%
%  untitled
%
%  Created by Etienne van Delden on 2008-07-03.
%  Copyright (c) 2008 __MyCompanyName__. All rights reserved.
%
\documentclass[]{article}

% Use utf-8 encoding for foreign characters
\usepackage[utf8]{inputenc}

% Setup for fullpage use
\usepackage{fullpage}

% Uncomment some of the following if you use the features
%
% Running Headers and footers
%\usepackage{fancyhdr}

% Multipart figures
%\usepackage{subfigure}

% More symbols
%\usepackage{amsmath}
%\usepackage{amssymb}
%\usepackage{latexsym}

% Surround parts of graphics with box
\usepackage{boxedminipage}

% Package for including code in the document
\usepackage{listings}

% If you want to generate a toc for each chapter (use with book)
\usepackage{minitoc}

% This is now the recommended way for checking for PDFLaTeX:
\usepackage{ifpdf}

%\newif\ifpdf
%\ifx\pdfoutput\undefined
%\pdffalse % we are not running PDFLaTeX
%\else
%\pdfoutput=1 % we are running PDFLaTeX
%\pdftrue
%\fi

\ifpdf
\usepackage[pdftex]{graphicx}
\else
\usepackage{graphicx}
\fi
\title{ 2IP15: Programming Methonds \\
Documentation for Assignment 3, part 3: FormulaLab and Undo/Redo}
\author{E.I.R. van Delden\\ student: 0618959}

\date{2008-07-03}

\begin{document}

\ifpdf
\DeclareGraphicsExtensions{.pdf, .jpg, .tif}
\else
\DeclareGraphicsExtensions{.eps, .jpg}
\fi

\maketitle


\begin{abstract}
  This document will give a, very, brief summary of my Undo/Redo Implementation for the FormulaLab program.
\end{abstract}

\section{Introduction}
I have added 2 new units, \emph{UndoRedo} and \emph{TreeEditorCommands}. These units are situated in \begin{verbatim}
  /Components/Base/UndoRedo.pas
\end{verbatim}
and \begin{verbatim}
  /Companents/Base/TreeEditorCommands.pas
\end{verbatim}
respectively.\\

\subsection{UndoRedo} % (fold)
\label{sub:undoredo}
The UndoRedo unit is exactly the same as seen in the Appendix of the UndoRedo note.
% subsection undoredo (end)

\subsection{TreeEditorCommands} % (fold)
\label{sub:treeeditorcommands}
TreeEditorCommands is based on the unit \emph{LineEditorCommands} from the extra UndoRedo note. 
The following assumption was made:
\begin{quote}
 \emph{ ``Using the Undo and Redo commands, do not effect the clipboard.''}
\end{quote}
This is a commonly made assumption, seen in more applications, like Microsoft Word or the texteditor used for this document (Macromates' TextMate).\\ \\

The following operations have been added to TreeEditorCommands, to support Undo-/Redoing:

\begin{itemize}
  \item DoClear
  \item DoCut
  \item DoDelete
  \item DoExpand
  \item DoPaste
\end{itemize}

Because of the above made assumption, DoCopy was ommitted. \\
Every command has an \emph{create, execute, reverse} and \emph{reversible} method.
\subsubsection{Create} % (fold)
\label{ssub:create}
 For each command the variable \emph{FTreeEditor} is given as an argument, so we can access the Tree form within \emph{command.execute}. Expand and Paste have an extra argument, for replacing a part of the tree.\\
 All commands have at least 1 private field, \emph{FDeletedTree}. FDeletedTree stores the old tree that's been deleted (the entire tree, not an subtree nor a pointer to a father). 
 
% subsubsection create (end)

\subsubsection{Execute} % (fold)
\label{ssub:execute}
The code for the old implementation is literally copied to the execute method and slightly adjusted. The assert's and all code for the clipboard was removed.\\

Preceding the copied code, there is an extra assignment 
\begin{verbatim}
  FDeletedTree := FTreeEditor;
\end{verbatim}
  So we don't lose any data when we want to undo the comman

% subsubsection execute (end)

\subsubsection{Reverse / Reversible} % (fold)
\label{ssub:reverse_reversible}

All given Commands change the tree in a way, so all commands are \emph{Reversible} (result = true).\\
Reverse has for all commands the following implementation:
\begin{verbatim}
    FTreeEditor.Replace(FDeletedTree, FTreeEditor.Tree);
\end{verbatim}
It restores to the tree, before the command was executed.

% subsubsection reverse_reversible (end)

% subsection treeeditorcommands (end)

\section{Implementation} % (fold)
\label{sec:implemantation}
An extra private variable has been added, FController: TController, that is created on Initialization. \\
For the 5 original commands, the asserts and all code with respect to the clipboard have been untouched. The real working code has been moved to corresponding execute command in the unit TreeEditorCommands. 

The implementations of the 5 original commands have been replaced with:
\begin{verbatim}
  Command := TCommand_DoClear.Create(FTreeEditor);
  FController.DoCommand(Command);
\end{verbatim}
The corresponding Command object is created and the fed to the Controller, that executes the command and adjusts the 2 stacks accordingly.\\
2 new buttons have been added, \emph{Undo} and \emph{Redo}. These buttons can be used to Undo or Redo a certain command. 
\\ \\Note: The current implementation isn't bug free. 1 command can succesfully be undone and redone, but troubles arise when trying to undo more commands. Part of it is, so I believe, because of the chosen implementation.
% section implementation (end)

\bibliographystyle{plain}
\bibliography{}
\end{document}
